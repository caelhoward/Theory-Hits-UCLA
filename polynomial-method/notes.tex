\documentclass[11pt]{article}

% Prefix for numedquestion's
\newcommand{\questiontype}{Question}


% Use this if your "written" questions are all under one section
% For example, if the homework handout has Section 5: Written Questions
% and all questions are 5.1, 5.2, 5.3, etc. set this to 5
% Use for 0 no prefix. Redefine as needed per-question.
\newcommand{\writtensection}{0}

\usepackage{amsmath, amsfonts, amsthm, amssymb}  % Some math symbols
\usepackage{mathtools}
\usepackage{algorithm}
\usepackage{algpseudocode}
\usepackage{dsfont}

\newtheorem{theorem}{Theorem}[section]
\theoremstyle{definition}
\newtheorem{definition}{Definition}[section]
\newtheorem{lemma}[theorem]{Lemma}

\usepackage{centernot}
\usepackage{mathtools}

\usepackage{enumitem}

\setlength{\parindent}{0pt}

\begin{document}

\subsection{Notes on Checksum Communication Complexity}
A paper has shown that the Number on Forehead communication complexity of CheckSum has been bounded as follows:
$$(\log N)^{\Omega(1)} \leq NOF(CheckSum) \leq O(\sqrt{\log N})$$
There are many different techniques to analyze something called Ramsey type Problems, where we try to maximize the size of a set while avoiding a certain pattern (like 3-Arithmetic Progressions).
\begin{itemize}
    \item Graph Theory
    \item Ergodic Theory
    \item Fourier Analysis (Most Success)
    \item Polynomial Method
\end{itemize}
\section{3AP Over Finite Field}
The 3AP problem is the same as the original problem over $[N]$, however, we focus on the case where arithmetic is done over $\mathbb{R}_3^n$. This argument generalizes to any group where there exists a plus operation and we make a set such that:
$$\nexists a,b,c | a \neq b \neq c, a + c = 2b$$
We want to determine, given the universe $U = \{0,1,2\}^n$ where addition is mod 3:
$$r_3(\mathbb{F}_3^n) = \max |S| \text{ where } S \subseteq \{0,1,2\}^n\text{ and no 3AP exists.}$$
This is also called the "CAP-SET PROBLEM".
\subsection{Size of CAP-SET}
Recall that Beherend's construction led to a subset $S \subseteq [N]$, where:
$$|S| \subseteq \frac{N}{2^{c\cdot \sqrt{\log N}}}$$
In CAPSET case, the size of $N$ is $3^n$. We argue that it is easier to form a $3AP$ in $\mathbb{F}^n_3$ than in $[3^n]$.
\bigskip

Intuition: Finding solutions to $x+y = 2z$ modular arithmetic is easier over $\mathbb{F}_3^n$ than $\mathbb{Z}$. ( I do not get his argument )

\begin{theorem}
    $r_3(\mathbb{F}_3^n) \leq (2.76)^n = N^c$ for some $c < 1$.
\end{theorem}
Recall that $N/(2^{c\cdot \sqrt{\log N}}) \leq r_3(\mathbb{Z}_n)$
\bigskip

\subsection{Examples and Properties}
Firstly, we know that we can select a set that has size $2^n$ that satisfies that there are no $3AP$'s. We argue that $A = \{0,1\}^n$ has no $3AP$'s.
\begin{proof}
Assume that there was some $a,b,c$ such that they form a nontrivial $3AP$. The argument is relatively straightforward, but we know that if $a_i + b_i$ is $1$, $c_i$ cannot be a three $AP$, if $a_i + b_i = 0$ then $a_i=b_i=c_i=0$ so they must be the same bit and if $a_i + b_i = 2$ then $c_i = b_i = a_i = 1$. So either the three numbers are the same or they do not form a $3AP$.
\end{proof}
This provides a lower bound for $r_3(\mathbb{F}_3^n)$:
$$r_3(\mathbb{F}_3^n) \geq 2^n = N^{\log_3 2} \approx N^{0.63}$$
What do $3AP$s over mod $3$ arithmetic look like? We know that:
$$(a,b,c) \in \{0,1,2\}^n \text{ and } a+b=2c$$
We can think of it as:
\begin{align*}
    a + b &\equiv 2c \mod 3\\
    \forall i, a_i + b_i &\equiv 2c_i \mod 3\\
    a_i + b_i &\equiv -c_i \mod 3\\
    a_i + b_i + c_i &\equiv 0 \mod 3
\end{align*}
This equation is satisifed when all $a_i = b_i = c_i$ or each value is unique.
\end{document}